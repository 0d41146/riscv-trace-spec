\chapter{Future Directions} \label{Future}

The current focus is the compressed branch trace, however there a
number of other types of processor trace that would be useful 
(detailed below in no particular order). These
should be considered as possible features that maybe added in the future,
once the current scope has been completed.

\section{Data trace}

The trace encoder will output packets to communicate information
about loads and stores to an off-chip decoder.  To reduce the amount
of bandwidth required, reporting data values will be optional, and
both address and data will be able to be encoded differentially when
it is beneficial to do so.  This entails outputting the difference
between the new value and the previous value of the same transfer
size, irrespective of transfer direction.

Unencoded values will be used for synchronisation and at other times.

\section{Fast profiling}

In this mode the encoder will provide a non-intrusive alternative to
the traditional method of profiling that requires the processor to
be halted periodically so that the program counter can be sampled.
The encoder will issue packets when an exception, call or return is
detected, to report the next instruction executed (i.e. the
destination instruction).  Optionally, the encoder will also be able to
report the current instruction (i.e. the source instruction).

\section{Inter-instruction cycle counts}

In this mode the encoder will trace where the hart is stalling by
reporting the number of cycles between successive instruction
retirements.

\section{Transport}

After the current charter has been satisfied the transport mechanism
should be defined and standardised. This will include Aurora based
serdes, PCIe and Ethernet.
