\chapter{Encoder Control} \label{encoderControl}

The fields required to control a Trace Encoder are defined in the 
\href{https://github.com/riscv-non-isa/tg-nexus-trace/blob/master/docs/RISC-V-Trace-Control-Interface.adoc}{RISC-V Trace Control Interface Specification},
which is intended to apply to any and all RISC-V trace encoders, regardless of encoding protocol.
This chapter details which of those fields apply to E-Trace.  To avoid replication, descriptions are not provided here; 
additional E-Trace specific context or clarification is provided only where required.

How fields are organized and accessed (e.g packet based or memory mapped) is outside the scope of this document.  
If a memory mapped approach is adopted, \href{https://github.com/riscv-non-isa/tg-nexus-trace/blob/master/docs/RISC-V-Trace-Control-Interface.adoc\#register-map}
{this register map} from the RISC-V Trace Control Interface Specification should be used.

Note: Upto and including the E-Trace v2.0.0 specification, which predated
the creation of the RISC-V Trace Control Interface Specification, the full field definitions 
were included in this chapter.  For versions later than this, the field definitions have simply moved from this
specification to the RISC-V Trace Control Interface Specification, without any change to their meaning.  However, in order to 
create a more widely applicable protocol agnostic specification it has been necessary to change the field names in the 
process.

The applicability of fields for E-trace is categorized as follows:

\begin{itemize}
  \item N: Not applicable
  \item M: Mandatory
  \item O: Optional
  \item MD: Mandatory if data trace is supported
  \item OD: Optional for data trace
\end{itemize}

\FloatBarrier
\section{Basic Control} \label{sec:ctl-basic}

The following fields control basic encoding behavior.

\begin{table}[htp]
  \centering
  \caption{Basic Control}
  \label{tab:ctl-basic}
  \begin{tabulary}{\textwidth}{|l|l|p{90mm}|}
    \hline
    {\bf Field} & {\bf Applicability} & {\bf E-Trace Specific Details} \\
    \hline
    \textbf{trTeActive} & M & \\
    \hline
    \textbf{trTeEnable} & M & \\
    \hline
    \textbf{trTeInstTracing} & M & \\
    \hline
    \textbf{trTeDataTracing} & MD & \\
    \hline
    \textbf{trTeInstTrigEnable} & O & \\
    \hline
    \textbf{trTeDataTrigEnable} & OD & \\
    \hline
    \textbf{trTeInstStallOrOverflow} & O & \\
    \hline
    \textbf{trTeDataStallOrOverflow} & OD & \\
    \hline
    \textbf{trTeInstStallEn} & O & \\
    \hline
    \textbf{trTeDataStallEn} & OD & \\
    \hline
    \textbf{trTeEmpty} & O & \\
    \hline
    \textbf{trTeDataDrop} & OD & \\
    \hline
    \textbf{trTeDataDropEn} & OD & \\
    \hline
    \textbf{trTeInhibitSrc} & O & \\
    \hline
    \textbf{trTeInstSyncMode} & M & If hardcoded, must be to a non-zero value. \\
    \hline
    \textbf{trTeInstSyncMax} & M & May be hardcoded.\\
    \hline
    \textbf{trTeFormat} & M & Must be set to 0 (denoting E-Trace format).\\
    \hline
    \textbf{trTeVerMajor} & M & \\
    \hline
    \textbf{trTeVerMinor} & M & \\
    \hline
    \textbf{trTeCompType} & M & \\
    \hline
    \textbf{trTeProtocolMajor} & M & Must be 0 to indicate this version (2.0.x) of the E-Trace protocol.\\
    \hline
    \textbf{trTeVerMinor} & M & Must be 0.\\
    \hline
    \textbf{trTeSrcID} & O & \\
    \hline
    \textbf{trTeSrcBits} & O & \\
    \hline
  \end{tabulary}
\end{table}

\section{Optional Modes} \label{sec:ctl-modes}

See section~\ref{optional} for details of the modes covered in this section.

\begin{table}[htp]
  \centering
  \caption{Optional and run-time configurable modes}
  \label{tab:ctl-resync}
  \begin{tabulary}{\textwidth}{|l|l|p{90mm}|}
    \hline
    {\bf Field} & {\bf Applicability} & {\bf E-Trace Specific Details} \\
    \hline
    \textbf{trTeInstNoAddrDiff} & O & \\
    \hline
    \textbf{trTeInstNoTrapAddr} & O & \\
    \hline
    \textbf{trTeInstEnSequentialJump} & O & \\
    \hline
    \textbf{trTeInstEnImplicitReturn} & O & \\
    \hline
    \textbf{trTeInstEnBranchPrediction} & O & \\
    \hline
    \textbf{trTeInstJumpTargetCache} & O & \\
    \hline
    \textbf{trTeDataNoValue} & OD & \\
    \hline
    \textbf{trTeDataNoAddr} & OD & \\
    \hline
    \textbf{trTeDataAddrCompress} & OD & \\
    \hline
    \textbf{trInstMode} & N & Hardcode to 7.\\
    \hline
    \textbf{trTeInstImplicitReturnMode} & N & Hardcode to 0.\\
    \hline
    \textbf{trTeInstEnRepeatedHistory} & N & Hardcode to 0.\\
    \hline
    \textbf{trTeInstEnAllJumps} & N & Hardcode to 0.\\
    \hline
    \textbf{trTeInstExtendAddrMSB} & N & Hardcode to 0.\\
    \hline
  \end{tabulary}
\end{table}

\section{Filtering} \label{sec:ctl-filter}

See section~\ref{ch:filtering} for details of the filtering capabilities covered in this section.

\begin{table}[htp]
  \centering
  \caption{Trace filtering selection}
  \label{tab:ctl-filtersel}
  \begin{tabulary}{\textwidth}{|l|l|p{90mm}|}
    \hline
    {\bf Field} & {\bf Applicability} & {\bf E-Trace Specific Details} \\
    \hline
    \textbf{trTeInstFilters} & O & \\
    \hline
    \textbf{trTeDataFilters} & OD & \\
    \hline
    \textbf{trTeFilter...} & O & \\
    \hline
    \textbf{trTeComp...} & O & \\
    \hline
    \textbf{trTeTrig...} & N & Hardcode to 0.\\
    \hline
  \end{tabulary}
\end{table}
