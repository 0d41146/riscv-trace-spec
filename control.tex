\chapter{Encoder Control} \label{encoderControl}

Fields are categorized into the following groups:

\begin{itemize}
  \item M: Mandatory
  \item O: Optional
  \item MD: Mandatory if data trace is supported
  \item OD: Optional for data trace
  \item F: Optional for filtering
\end{itemize}

Other abbreviations used in the tables are:
\begin{itemize}
  \item \textbf{W} column heading indicates field width (in bits)
  \item \textbf{G} column heading indicates field group
  \item \textbf{RW} column heading indicates whether bit is read-only (R), or read-write (RW).  
    For the latter, it is allowed but not required that the bit be writable
  \item \textbf{Rst} column heading indicates field reset value. SD in this column indicates a system dependent reset value
\end{itemize}

\section{Basic Control} \label{sec:ctl-basic}

The following fields control basic encoding behavior.

\begin{table}[htp]
  \centering
  \caption{Basic Control}
  \label{tab:ctl-basic}
  \begin{tabulary}{\textwidth}{|l|l|l|l|l|p{90mm}|}
    \hline
    {\bf Field} & {\bf W} & {\bf G} & {\bf RW} & {\bf Rst} & {\bf Description} \\
    \hline
    \textbf{Active} & 1 & M & RW & 0 & Master enable for trace system.  When 0, the trace system may have clocks gated off or be powered down,
     and other register locations may be inaccessible.
      Hardware may take arbitrarily long to process power-up or power-down and will indicate completion when the read value of this bit
      matches the value written.\\
    \hline
    \textbf{teEnable} & 1 & M & RW & 0 & 
      Master trace enable.  Trace can be enabled via \textbf{iTracing}
      or \textbf{dTracing} when 1. Setting to 0 flushes any queued trace data to the designated sink.\\
    \hline
    \textbf{iTracing} & 1 & M & RW & 0 & Instruction trace enable.  When 1, trace will be generated,
       subject to any optional filtering.  May be written by software, or via triggers if \textbf{iTrigEnable} is 1.  \\
    \hline
    \textbf{dTracing} & 1 & MD & RW & 0 & Data trace enable.  When 1, trace will be generated,
       subject to any optional filtering.  May be written by software, or via triggers if \textbf{dTrigEnable} is 1.  \\
    \hline
    \textbf{iTrigEnable} & 1 & O & RW & 0 & When 1, allows iTracing to be set or cleared by \textit{trace-on} and \textit{trace-off} 
      Debug module triggers respectively (see \ref{sec:trigger}).\\
    \hline
    \textbf{dTrigEnable} & 1 & OD & RW & 0 & When 1, allows dTracing to be set or cleared by \textit{trace-on} and \textit{trace-off} 
      Debug module triggers respectively (see \ref{sec:trigger}).\\
    \hline
    \textbf{stallEnable} & 1 & O & RW & 0 & When 0, if the encoder cannot accept trace input from the RISC-V hart, trace is lost, and is
      indicated via the \textit{Support} trace packet (see \textcolor{green}{section}~\ref{sec:format33}).\newline
      When 1, the \textbf{stall} output signal is asserted to stall the RISC-V hart until trace can be accepted (see table~\ref{tab:egress-side-band}).\\
    \hline
    \textbf{Empty} & 1 & O & R & 1 & Reads as 1 when all trace has been emitted.  Note: this status is also indicated via the 
      \textit{Support} trace packet (see table~\ref{tab:te_inst3-3}).\\
    \hline
    \textbf{ResyncMode} & 2 & M & RW & SD & Selects the resynchronization mechanism.  At least one non-zero mechanism must be implemented.\newline
    0: Off\newline
    1: Count trace packets\newline
    2: Count clock cycles\newline
    3: Count instruction (16-bit) half-words\\
    \hline
    \textbf{ResyncMax} & 4 & O & RW & SD & The maximum interval (in units determined by \textbf{ResyncMode}) between synchronization packets
    (see \textcolor{green}{sections}~\ref{sec:format30} and~\ref{sec:format31}) is 2\textsuperscript{\textbf{ResyncMax} + 4}.\\
    \hline
  \end{tabulary}
\end{table}

\FloatBarrier
\section{Optional Modes} \label{sec:ctl-modes}

See section~\ref{optional} for details of the modes covered in this section.

\begin{table}[htp]
  \centering
  \caption{Optional and run-time configurable modes}
  \label{tab:ctl-resync}
  \begin{tabulary}{\textwidth}{|l|l|l|l|l|p{80mm}|}
    \hline
    {\bf Field} & {\bf W} & {\bf G} & {\bf RW} & {\bf Rst} & {\bf Description} \\
    \hline
    \textbf{FullAddress} & 1 & O & RW & SD & Send only full (non-differential) addresses when set\\
    \hline
    \textbf{ImplicitExcept} & 1 & O & RW & SD & When set, do not send exception address, only exception cause in 
      Exception packets (see table~\ref{sec:format31})\\
    \hline
    \textbf{siJump} & 1 & O & RW & SD & Do not treat sequentially inferrable jumps as uninferable PC discontinuities when set.\\
    \hline
    \textbf{ImplicitReturn} & 1 & O & RW & SD & Do not treat returns as uninferable PC discontinuities when set.\\
    \hline
    \textbf{BranchPrediction} & 1 & O & RW & SD & Branch predictor enabled when set.\\
    \hline
    \textbf{JumpTargetCache} & 1 & O & RW & SD & Jump target cache enabled when set.\\
    \hline
    \textbf{noData} & 1 & OD & RW & SD & Omit data from data trace packets when set.\\
    \hline
    \textbf{noAddress} & 1 & OD & RW & SD & Omit address from data trace packets when set.\\
    \hline
    \textbf{DataCompress} & 2 & OD & RW & SD & Data trace compression selection:\newline
    0: Only send full (unmodified) addresses\newline
    1: Use XOR compression\newline
    2: Use differential compression\newline
    3: Dynamically select XOR or differential on a per-packet basis in order to minimize packet length\\
    \hline
  \end{tabulary}
\end{table}


\section{Filtering} \label{sec:ctl-filter}

See section~\ref{ch:filtering} for details of the filtering capabilities covered in this section.

The fields in table~\ref{tab:ctl-filtersel} determine how filtering is applied to instruction and data trace.  These are an extension of
the basic controls in table~\ref{tab:ctl-basic}.  Grouping all filters together as a single vector makes it possible to 
enable or change the filtering selection of multiple filters simultaneously.

\begin{table}[htp]
  \centering
  \caption{Trace filtering selection}
  \label{tab:ctl-filtersel}
  \begin{tabulary}{\textwidth}{|l|l|l|l|l|p{90mm}|}
    \hline
    {\bf Field} & {\bf W} & {\bf G} & {\bf RW} & {\bf Rst} & {\bf Description} \\
    \hline
    \textbf{iFilters} & 16 & O & RW & 0 & Determine which filters qualify instruction trace.  If bit \textit{N} is a 1 then instructions will be traced
      when filter \textit{N} matches.  If all bits are 0, all instructions are traced.\\
    \hline
    \textbf{dFilters} & 16 & OD & RW & 0 & Determine which filters qualify data trace.  If bit \textit{N} is a 1 then data accesses will be traced
      when filter \textit{N} matches.  If all bits are 0, all data accesses are traced.\\
    \hline
  \end{tabulary}
\end{table}

The fields in table~\ref{tab:ctl-filter} must be replicated for each filter, and determine which HART inputs the filter is 
sensitive to.

\begin{table}[htp]
  \centering
  \caption{Filter Control}
  \label{tab:ctl-filter}
  \begin{tabulary}{\textwidth}{|l|l|l|l|l|p{80mm}|}
    \hline
    {\bf Field} & {\bf W} & {\bf G} & {\bf RW} & {\bf Rst} & {\bf Description} \\
    \hline
    \textbf{filterEnable} & 1 & O & RW & 0 & Overall filter enable.\\
    \hline
    \textbf{matchComp1} & 1 & O & RW & 0 & When set, the output of the comparator selected by \textbf{comp1} must be high in order for the filter to match.\\
    \hline
    \textbf{comp1} & 3 & O & RW & 0 & Specifies the comparator unit to use for the 1st comparison.\\
    \hline
    \textbf{matchComp2} & 1 & O & RW & 0 & When set, the output of the comparator selected by \textbf{comp2} must be high in order for the filter to match.\\
    \hline
    \textbf{comp2} & 3 & O & RW & 0 & Specifies the comparator unit to use for the 2nd comparison.\\
    \hline
    \textbf{matchComp2} & 1 & O & RW & 0 & When set, the output of the comparator selected by \textbf{comp3} must be high in order for the filter to match.\\
    \hline
    \textbf{comp2} & 3 & O & RW & 0 & Specifies the comparator unit to use for the 3rd comparison.\\
    \hline
    \textbf{matchPrivilege} & 1 & O & RW & 0 & When set, match privilege levels specified by \textbf{matchChoicePrivilege}.\\
    \hline
    \textbf{matchEcause} & 1 & O & RW & 0 & When set, start matching from exception cause codes specified by \textbf{matchChoiceEcause}, 
      and stop matching upon return from the 1st matching exception.\\
    \textbf{matchInterupt} & 1 & O & RW & 0 & When set, start matching from a trap with the interrupt level codes specified by \textbf{matchValueInterrupt}, 
      and stop matching upon return from the 1st matching trap.\\
    \hline
    \textbf{matchImpdef} & 1 & O & RW & 0 & When set, match \textbf{impdef} values as specified by \textbf{matchValueImpdef} and \textbf{matchMaskImpdef}.\\
    \hline
    \textbf{matchDtype} & 1 & OD & RW & 0 & When set, match \textbf{dtype} values as specified by \textbf{matchChoiceDtype}.\\
    \hline
    \textbf{matchDsize} & 1 & OD & RW & 0 & When set, match \textbf{dsize} values as specified by \textbf{matchChoiceDsize}.\\
    \hline
  \end{tabulary}
\end{table}

The fields in table~\ref{tab:ctl-match} must be replicated for each filter, and determine the values the
filter must match for the HART inputs enabled via the fields in table~\ref{tab:ctl-filter}.  

\begin{table}[htp]
  \centering
  \caption{Filter Match Control}
  \label{tab:ctl-match}
  \begin{tabulary}{\textwidth}{|l|l|l|l|l|p{80mm}|}
    \hline
    {\bf Field} & {\bf W} & {\bf G} & {\bf RW} & {\bf Rst} & {\bf Description} \\
    \hline
    \textbf{matchChoicePrivilege} & 8 & O & RW & SD & When \textbf{matchPrivilege} is set, match all privilege levels for which the corresponding bit is set.
       For example, if bit \textit{N} is 1, then match if \textcolor{green}{the \textbf{priv} value is \textit{N} (see table~\ref{tab:priv} for privilege level encoding).}\\
    \hline
    \textbf{matchValueInterrupt} & 1 & O & RW & SD & When \textbf{matchinterrupt} is set, match \textbf{itype} of 2 or 1 depending on whether this bit is 
       1 or 0 respectively.\\
    \hline
    \textbf{matchChoiceEcause} & 32 & O & RW & SD & When \textbf{matchEcause} is set, match all excepion causes for which the corresponding bit is set.
       For example, if bit \textit{N} is 1, then match if the cause is \textit{N}.\\
    \hline
    \textbf{matchMaskImpdef} & See\textsuperscript{1} & O & RW & SD & When \textbf{matchImpdef} is set, match if \newline
       (\textbf{impdef} \& \textbf{matchMaskImpdef}) == (\textbf{matchValueImpdef} \& \textbf{matchMaskImpdef})\\
    \hline
    \textbf{matchValueImpdef} & See\textsuperscript{1} & O & RW & SD & \\
    \hline
    \textbf{matchChoiceDtype} & 16 & O & RW & SD & When \textbf{matchDtype} is set, match all data access types for which the corresponding bit is set.
       For example, if bit \textit{N} is 1, then match if the \textbf{dtype} is \textit{N}.\\
    \hline
    \textbf{matchChoiceDsize} & 8 & O & RW & SD & When \textbf{matchDsize} is set, match all data access sizes for which the corresponding bit is set.
       For example, if bit \textit{N} is 1, then match if the \textbf{dsize} is \textit{N}.\\
    \hline
    \multicolumn{6}{|l|}{1: \textit{impdef\_width\_p}}\\
    \hline
  \end{tabulary}
\end{table}

The fields in tables~\ref{tab:ctl-comp} must be replicated for each comparator.

\begin{table}[htp]
  \centering
  \caption{comparator Control}
  \label{tab:ctl-comp}
  \begin{tabulary}{\textwidth}{|l|l|l|l|l|p{90mm}|}
    \hline
    {\bf Field} & {\bf W} & {\bf G} & {\bf RW} & {\bf Rst} & {\bf Description} \\
    \hline
    \textbf{pInput} & 2 & O & RW & SD & Determines which input bus to compare against the primary comparator\newline
    0: \textbf{iaddr}\newline
    1: \textbf{context}\newline
    2: \textbf{tval}\newline
    3: \textbf{daddr}\\
    \hline
    \textbf{sInput} & 2 & O & RW & SD & Determines which input bus to compare against the secondary comparator.  Encoding as per \textbf{pInput}\\
    \hline
    \textbf{pFunction} & 3 & O & RW & SD & Selects the primary comparator function\newline
    0: selected input equal to \textbf{pMatch}\newline
    1: selected input not equal to \textbf{pMatch}\newline
    2: selected input less than \textbf{pMatch}\newline
    3: selected input less than or equal to \textbf{pMatch}\newline
    4: selected input greater than \textbf{pMatch}\newline
    5: selected input greater than or equal to \textbf{pMatch}\newline
    6: reserved (always match)\newline
    7: always match\\
    \hline
    \textbf{sFunction} & 3 & O & RW & SD & Selects the secondary comparator function\newline
    0: selected input equal to \textbf{sMatch}\newline
    1: selected input not equal to \textbf{sMatch}\newline
    2: selected input less than \textbf{sMatch}\newline
    3: selected input less than or equal to \textbf{sMatch}\newline
    4: selected input greater than \textbf{sMatch}\newline
    5: selected input greater than or equal to \textbf{sMatch}\newline
    6: use \textbf{sMatch} as a mask for \textbf{pMatch} (always match)\newline
    7: always match\\
    \hline
    \textbf{matchMode} & 2 & O & RW & SD & Determines which input bus to compare against the primary comparator\newline
    0: primary function matches\newline
    1: primary and secondary functions both match: (P\&\&S)\newline
    2: either primary or secondary function doesn't match: !(P\&\&S)\newline
    3: set when primary matches and continue to match until after secondary matches\\
    \hline
    \textbf{pMatch} & See\textsuperscript{1} & O & RW & SD & The match value for the primary comparator\\
    \hline
    \textbf{sMatch} & See\textsuperscript{1} & O & RW & SD & The match value for the secondary comparator\\
    \hline
    \textbf{pNotify} & 1 & O & RW & 0 & Generate a trace packet explicitly reporting the address of the final instruction in a block
      that causes a primary match (requires \textbf{pInput} to be 0).  This is also known as a watchpoint.\\
    \hline
    \textbf{sNotify} & 1 & O & RW & 0 & Generate a trace packet explicitly reporting the address of the final instruction in a block
      that causes a secondary match (requires \textbf{sInput} to be 0).  This is also known as a watchpoint.\\
    \hline
    \multicolumn{6}{|l|}{1: Max of supported input widths \textit{iaddress\_width\_p}, \textit{context\_width\_p}, \textit{daddress\_width\_p}}\\
    \hline
  \end{tabulary}
\end{table}

\FloatBarrier
\section{Buffering and Transport} \label{sec:ctl-transport}

Details to follow

