\chapter{Encoder Control} \label{encoderControl}

Need to add definitions for individual fields here.

Fields are categorized into the following groups:

\begin{itemize}
  \item M: Mandatory
  \item O: Optional
  \item MD: Mandatory if data trace is supported
  \item OD: Optional for data trace
  \item F: Optional for filtering.
\end{itemize}

Other abreviations used in the tables are:
\begin{itemize}
  \item \textbf{W} column heading indicates field width
  \item \textbf{G} column heading indicates field group
  \item \textbf{Rst} column heading indicates field reset value
  \item SD in \textbf{Rst} column indicates a system dependent reset value
\end{itemize}

\section{Basic Control} \label{sec:ctl-basic}

The following fields control basic encoding behaviour.

\begin{table}[htp]
  \centering
  \caption{Basic Control}
  \label{tab:ctl-basic}
  \begin{tabulary}{\textwidth}{|l|l|l|l|l|p{90mm}|}
    \hline
    {\bf Field} & {\bf W} & {\bf G} & {\bf RW} & {\bf Rst} & {\bf Description} \\
    \hline
    \textbf{Active} & 1 & M & RW & 0 & Master enable for trace system.  When 0, the trace system may have clocks gated off or be powered down,
     and other register locations may be inaccessible.
      Hardware may take arbitrarily long to process power-up or power-down and will indicate completion when the read value of this bit
      matches the value written.\\
    \hline
    \textbf{iEnable} & 1 & M & RW & 0 & Instruction trace enable.  Setting to 0 flushes any queued instruction trace.\\
    \hline
    \textbf{dEnable} & 1 & MD & RW & 0 & Data trace enable.  Setting to 0 flushes any queued data trace.\\
    \hline
    \textbf{iTrigEnable} & 1 & O & RW & 0 & When 1, allows \textbf{iEnable} to be set or cleared by \textit{trace-on} and \textit{trace-off} 
      Debug module triggers respectively (see \ref{sec:trigger}).\\
    \hline
    \textbf{dTrigEnable} & 1 & OD & RW & 0 & When 1, allows \textbf{dEnable} to be set or cleared by \textit{trace-on} and \textit{trace-off} 
      Debug module triggers respectively (see \ref{sec:trigger}).\\
    \hline
    \textbf{stallEnable} & 1 & O & RW & 0 & When 0, if the encoder cannot accept trace input from the RISC-V hart, trace is lost, and is
      indicated via the \textit{Support} trace packet (see table~\ref{sec:format31}).\newline
      When 1, the \textbf{stall} output signal is asserted to stall the RISC-V hart until trace can be accepted (see table~\ref{tab:egress-side-band}).\\
    \hline
    \textbf{Empty} & 1 & O & R & 1 & Reads as 1 when all trace has been emitted.  Note: this status is also indicated via the 
      \textit{Support} trace packet (see table~\ref{sec:format31}).\\
    \hline
    \textbf{ResyncMode} & 2 & M & RW & SD & Selects the resynchronization mechanism.  At least one non-zero mechanism must be implemented.\newline
    0: Off\newline
    1: Count trace packets\newline
    2: Count clock cycles\newline
    3: Count instruction half-words\\
    \hline
    \textbf{ResyncMax} & 4 & O & RW & SD & The maximum interval (in units determined by \textbf{ResyncMode}) between synchronization packets
    (see tables~\ref{sec:format30} and~\ref{sec:format31}) is 2\textsuperscript{\textbf{ResyncMax} + 4}.\\
    \hline
  \end{tabulary}
\end{table}

\FloatBarrier
\section{Optional Modes} \label{sec:ctl-modes}

See section~\ref{optional} for details of the modes covered in this section.  \textbf{Author note:} data trace mode descriptions need adding.

\begin{table}[htp]
  \centering
  \caption{Optional and run-time configurable modes}
  \label{tab:ctl-resync}
  \begin{tabulary}{\textwidth}{|l|l|l|l|l|p{90mm}|}
    \hline
    {\bf Field} & {\bf W} & {\bf G} & {\bf RW} & {\bf Rst} & {\bf Description} \\
    \hline
    \textbf{FullAddress} & 1 & O & RW & SD & Send only full (non-differential) addresses when set\\
    \hline
    \textbf{ImplicitExcept} & 1 & O & RW & SD & When set, do not send exception address, only exception cause in 
      Exception packets (see table~\ref{sec:format31})\\
    \hline
    \textbf{siJump} & 1 & O & RW & SD & Do not treat sequentially inferrable jumps as uninferable PC discontinuities when set.\\
    \hline
    \textbf{ImplicitReturn} & 1 & O & RW & SD & Do not treat returns as uninferable PC discontinuities when set.\\
    \hline
    \textbf{BranchPrediction} & 1 & O & RW & SD & Branch predictor enabled when set.\\
    \hline
    \textbf{JumpTargetCache} & 1 & O & RW & SD & Jump target cache enabled when set.\\
    \hline
  \end{tabulary}
\end{table}


\section{Filtering} \label{sec:ctl-filter}

Details to follow

\section{Buffering and Transport} \label{sec:ctl-transport}

Details to follow

\section{Message mapping to register proposal} \label{sec:messageToRegiser}

\begin{table}[htp]
  \tiny
  \centering
  \caption{Mapping UST messages to Register based control used in Nexus}
  \label{tab:te_control}
  \begin{tabulary}{\textwidth}{|l|l|p{15mm}|l|p{10mm}|l|p{15mm}|p{15mm}}

    \hline
    {\bf Field} & {\bf W} & {\bf Description} & {\bf Register} & {\bf Offset} & {\bf Message} & {\bf Offset} & {\bf UST Name} \\
    \hline
    \textbf {teActive} & 1 & Master Enable & \textbf {teControl} & 0 & \textbf {set\_enabled} & 16 & \textbf {module\_enable}\\
    \hline  
    \textbf {teEnable} & 1 & Trace Enable & \textbf{teControl} & 1 & \textbf {set\_trace.instruction} & 8 & \textbf {trace\_enable}\\     
    \hline
    \textbf{teTriggered} & 1 & Enable external triggers to change teEnable & \textbf {teControl} & TBD & \textbf {set\_trace.instruction} & 9 & \textbf {trigger\_enable}\\
    \hline
    \textbf{teTracing} & 1 & trace being generated & \textbf{teControl} & 2 & & \\
    \hline
    \textbf{teEmpty} & 1 & All trace emitted & \textbf{teControl} & 3 & & & Not necessary for RISC-V WG trace, as \textbf {te\_inst} support message issued after all trace emitted.\\
    \hline
    \textbf{teInstruction} & ? & trace mode enables & teControl & 4 & \textbf{set\_trace.instruction} & 37\newline
    38\newline
    39\newline
    40\newline
    41\newline
    42\newline
    43\newline & 
    \textbf {encoder\_mode} (2 bits)\newline
    \textbf{full\_address}\newline
    \textbf{implicit\_return}\newline
    \textbf{implicit\_except}\newline
    \textbf{sijump}\newline
    \textbf{branch\_prediction}\newline
    \textbf{jump\_target\_cache}\newline
    (others for proprietary extensions)\\
    \hline
    \textbf{teStallEnable} & 1 & Enable lossless behaviour & \textbf{teControl} & 13 & \textbf{set\_trace.instruction} & 35 & lossless\\
    \hline
    \textbf{teStallOnWrap} & 1 & Circular buffer control & \textbf{teControl} & 14	& - & \\
    \hline
    \textbf{teInhibitSrc} & 1 & Inhibit SrcID & \textbf{teControl} & 15 & - & &  Inappropriate in a message based architecture \\
    \hline
    \textbf{teSyncMax} & 4 & Resync interval & \textbf{teControl} & 16 & \textbf{set\_trace.instruction} & 13 & \textbf{max\_resync}\\
    \hline
    \textbf{teSyncMode} & 2 &Resync mode & \textbf{teControl} & TBD & \textbf{set\_trace.instruction} & 17 & \textbf{resync\_mode} \\
    \hline
    \textbf{teMessageFormat} & 3 & Trace format & \textbf{teControl} & 26 & - & & \\
    \hline
    \textbf{teSink} & 3 & Sink selector & \textbf{teControl} & 28 & - & UST trace routing is distributed and control mechanism is incompatible with this.\\
    \hline
  \end{tabulary}
\end{table}



\begin{table}[htp]
  \tiny
  \centering
  \caption{Mapping UST messages to Register based teImpl regster used in Nexus}
  \label{tab:te_control}
  \begin{tabulary}{\textwidth}{|l|l|p{15mm}|l|p{10mm}|l|p{15mm}|p{15mm}}
    \hline
    {\bf Field} & {\bf W} & {\bf Description} & {\bf Register} & {\bf Offset} & {\bf Message} & {\bf Offset} & {\bf UST Name} \\
    \hline
    \textbf{teFilter} & 16 & I-trace filter enable vector & TBD & TBD & \textbf{set\_trace.instruction} & 19 & \textbf{trigger} \\
    \hline
    \textbf{version} & 4 & TE version & \textbf{teImpl} & 0 & \textbf{discovery\_response} & TBD & arch \\
    \hline
    \textbf{hasSRAMSink} & 1 & TE has SRAM sink & \textbf{teImpl} & 4 & - & \\
    \hline
    \textbf{hasATBSink} & 1 & TE has ATB sink & \textbf{teImpl} & 5 & - &\\
    \hline
    \textbf{hasPIBSink} & 1 & TE has PIB sink & \textbf{teImpl} & 6 & - &\\
    \hline
    \textbf{hasSBASink} & 1 & TE has SBA sink & \textbf{teImpl} & 7 & - & &\\
    \hline
    \textbf{hasFunnelSink} & 1 & TE has funnel sink & \textbf{teImpl} & 8 & - &\\
    \hline
    \textbf{reserved} & 11 & 9 & & & & &\\
    \hline
    \textbf{SrcID} & 4 & Source ID & \textbf{teImpl} & 20 & - &\\
    \hline
    \textbf{nSrcBits} & 3 & Width of source ID & \textbf{teImpl} & 24 & & &\\
    \hline
  \end{tabulary}
\end{table}

