\chapter{Branch Trace} \label{Branch Trace}


Instruction delta tracing, also known as branch tracing, works by
tracking execution from a known start address by sending information
about the deltas taken by the program. Deltas are typically introduced
by jump, call, return and branch type instructions, although
interrupts and exceptions are also types of deltas.

Instruction delta tracing provides an efficient encoding of an
instruction sequence by exploiting the deterministic way the processor
behaves based on the program it is executing.

The approach relies on
an offline copy of the program binary being available to the decoder, so it
is generally unsuitable for either dynamic (self-modifying) programs
or those where access to the program binary is prohibited.

While the program binary is sufficient, access to the assembly or
higher-level source code will improve the ability of the decoder to present
the decoded trace in the debugger by annotating the traced instructions with
source code line numbers and labels, variable names etc.

This approach can be extended to cope with small sections of
deterministically dynamic code by arranging for the decoder to request
instruction memory from the target. Memory lookups generally lead to a
prohibitive reduction in performance, although they are suitable for
examining modest jump tables, such as the exception/interrupt vector
pointers of an operating system which may be adjusted at boot up and
when services are registered.  Both static and dynamically linked
programs can be traced using this approach. Statically linked programs
are straightforward as they generally operate in a known address
space, often mapping directly to physical memory. Dynamically linked
programs require the debugger to keep track of memory allocation
operations using either trace or stop-mode debugging.

\section{Instruction Delta Trace Concepts} \label{Trace Concepts}

\subsection{Sequential Instructions} \label{Sequential Instructions}

For instruction set architectures such as RISC-V where all instructions are executed
unconditionally or at least their execution can be determined based on
the program binary, the instructions between the deltas are assumed to be
executed sequentially. Consequently, there is no need
to report them in the trace. The trace only needs to contain whether
branches were taken or not, the addresses of taken indirect jumps, or
other program counter discontinuities.

\subsection{Uninferable PC Discontinuities} \label{uninfpc}

An uninferable program counter discontinuity is a program counter change
that can not be inferred from the program binary alone. For these cases,
the instruction delta trace must include a destination address: the
address of the next valid instruction.

Examples of this are indirect jumps, where
the next instruction address is determined by the contents of a
register rather than a constant embedded in the program binary.

\subsection{Branches} \label{branches}

A branch is an instruction where a jump is conditional on the
value of a register or a flag. For a decoder to able to follow program flow,
the trace must include whether a branch was taken or not.

For a direct branch, where the destination address is a constant that
is encoded in the program binary, no further information is required.
Direct branches are the only type of branch that is supported by the
RISC-V ISA.

\subsection{Interrupts and Exceptions} \label{interruptsexceptions}

Interrupts are a different type of delta that generally occur
asynchronously to the program's execution rather than intentionally as
a result of a specific instruction or event. Exceptions can be thought
of in the same way, even though they can be typically linked back to a
specific instruction address.

The decoder generally does not know
where an interrupt occured in the instruction sequence, so the trace
must report the address where normal program flow ceased, as well as
give an indication of the asynchronous destination which may be as
simple as reporting the exception type.  When an interrupt or
exception occurs, or the processor is halted, the final instruction
executed beforehand must be traced.  Following this, for an interrupt
or exception, the next valid instruction address (the first of the
interrupt or exception handler) must be traced in order to instruct the
trace decoder to classify the instruction as an indirect jump even
if it is not.

\subsection{Synchronization} \label{synchronization}

In order to make the trace robust there need to be regular
synchronization points within the trace. Synchronization is made by
sending a full valued instruction address (and potentially a context
identifier). The decoder and debugger may also benefit from sending
the reason for synchronising. The frequency of synchronization is a
trade-off between robustness and trace bandwidth.

The instruction trace encoder needs to synchronise fully:

\begin{itemize}

\item After a reset.
  \item When tracing starts.
\item If the instruction is the first of an interrupt service routine or
exception handler (hardware context change).
\item After a prolonged period of time.
\end{itemize}

\section{Optional and run-time configurable modes} \label{optional}

An instruction trace encoder may support multiple tracing modes.
To ensure that the decoder treats the incoming packets
correctly, it needs to be informed of the current active configuration.
The configuration is reported by a packet that is issued by the encoder
whenever the encoder configuration is changed.

Here are common examples of such modes:

\begin{itemize}
  \item delta address mode:
    program counter discontinuities are encoded as differences instead of absolute address values.
  \item full address mode:
    program counter discontinuities are encoded as absolute address values.
  \item implicit exception mode:
    the destination address of an exception (i.e. the address of the exception trap) is assumed to 
    be known by the decoder, and thus not encoded in the trace.
  \item implicit return mode:
    the destination address of function call returns is derived from a call stack, and thus not encoded
    in the trace.
  \item branch prediction mode:
    branches that are predicted correctly by an encoder branch predictor (and an identical copy in the decoder)
    are not encoded as taken/non-taken, but as a more efficient branch count number.
\end{itemize}

Modes may have associated parameters; see Table~\ref{tab:parameters} for further details.

\subsection{Delta address mode} \label{sec:delta-address}

Related parameters: None

In delta address mode, non-sequential program counter changes
are encoded as the difference between the previous and the current
address. This kind of encoding require less bits than encoding the
full address, and thus results in higher compression ratios.

This is the default encoding mode.

\subsection{Full address mode} \label{sec:full-address}

Related parameters: None

In full address mode, all addresses in the trace are encoded as absolute addresses instead
of in differential form. This kind of encoding is always less efficient, but it can be a useful 
debugging aid for software decoder developers.

\subsection{Implicit exception mode} \label{sec:implicit-exception}

Related parameters: None

The RISC-V Privileged ISA specification stores exception handler base
addresses in the \textbf{\textit{utvec/stvec/mvec}} CSR registers.
In some RISC-V implementations, the lower address bits are stored in
the \textbf{\textit{ucause/scause/mcause}} CSR registers.

By default, both the vec and cause values are reported when an exception or interrupt occurs.

The implicit exception mode omits vec, the trap handler address, from the trace and
thus improves efficiency.

This mode can only be used if the decoder can infer the address of the trap handler
from just the exception cause.

\subsection{Implicit return mode} \label{sec:implicit-return}

Related parameters: \textit{call\_counter\_size\_p}, \textit{return\_stack\_size\_p}.

Although a function return is usually an indirect jump, well behaved programs return to the
point in the program from which the function was called using a standard calling convention.
For those programs, it is possible to determine the execution path without being explicitly notified
of the destination address of the return.  The implicit return mode can result in very
significant improvements in trace encoder efficiency.

Returns can only be treated as inferable if the associated call has already been reported in
an earlier packet.  The encoder must ensure that this is the case.  This can be accomplished
by utilizing a counter to keep track of the number of nested calls being traced.  The counter
increments on calls (but not tail calls), and decrements on returns (see Section~\ref{Jump Classes}
for definitions).  The counter will not over or underflow, and is reset to 0 whenever a
synchronization packet is sent.  Returns will be treated as inferable and will not generate a trace
packet if the count is non-zero (i.e. the associated call was already reported in an earlier packet).

Such a scheme is low cost, and will work as long as programs are "well behaved".  The encoder does not check that the
return address is actually that of the instruction following the associated call.  As such, any program that
modifies return addresses cannot be traced using this mode with this minimal implementation.

Alternatively, the encoder can maintain a stack of expected return addresses, and only treat a
return as inferable if the actual return address matches the prediction.  This is fully robust for all
programs, but is more expensive to implement.  In this case, if a return address does not match the prediction, 
it must be reported explicitly via a packet, along with the number of return addresses
currently on the stack.  This ensures that the decoder can determine which return is being reported. 

\subsection{Branch prediction mode} \label{sec:branch-prediction}

Related parameters: \textit{bpred\_size\_p}.

In a straightforward encoder, the outcome of each executed branch is stored in
a branch map: a bit vector in which the taken/non-taken status of each branch is stored in
chronological order.

While this encoding is efficient, at 1 bit per branch, there are some cases where this
can still result in a relatively large volume of trace packets.  For example:

\begin{itemize}
  \item Executing tight loops of straight-line code.  Each iteration of the loop will add a bit to the branch map;
  \item Sitting in an idle loop waiting for an interrupt.  This produces large amounts of trace when nothing of 
  any interest is actually happening!  
  \item Breakpoints, which in some implementations also spin in an idle loop.
\end{itemize}

A significant coding efficiency can be obtained by the addition of a branch predictor in the encoder. To keep
the encoder and decoder synchronized, a predictor with identical behavior will need to be modelled in the decoder
software.

The predictor shall comprise a lookup table of 2\textsuperscript{N} entries, where N is specified by a parameter.  
Each entry is indexed by bits N:1 of the instruction address (or N+1:2 if compressed instructions aren't supported), 
and each contains a 2-bit prediction state:
\begin{itemize}
  \item 00: predict 0, transition to 01 if prediction fails;
  \item 01: predict 0, transition to 00 if prediction succeeds, else 11;
  \item 11: predict 1, transition to 10 if prediction fails;
  \item 10: predict 1, transition to 11 if prediction succeeds, else 00.
\end{itemize}

Other predictors, such as the gShare predictor (see Hennessy \& Patterson), should be considered.  Some further
experimentation is needed to determine the benefits of different lookup table sizes and predictor algorithms.

The lookup table entries are initialized to 00 when a synchronization packet is sent.

