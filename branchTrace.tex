\chapter{Branch Trace} \label{Branch Trace}


\section{Instruction Delta Tracing} \label{Delta Tracing}

Instruction delta tracing, also known as branch tracing, works by
tracking execution from a known start address by sending information
about the deltas taken by the program. Deltas are typically introduced
by jump, call, return and branch type instructions, although
interrupts and exceptions are also types of delta.

Instruction trace delta modes provide an efficient encoding of an
instruction sequence by exploiting the deterministic way the processor
behaves based on the program it is executing. The approach relies on
an offline copy of the program being available to the decoder, so it
is generally unsuitable for either dynamic (self-modifying) programs
or those where access to the program binary is prohibited. There is no
need for either assembly or high-level source code to be available,
although such source code will aid the debugger in presenting the
decoded trace.

This approach can be extended to cope with small sections of
deterministically dynamic code by arranging for the decoder to request
instruction memory from the target. Memory lookups generally lead to a
prohibitive reduction in performance, although they are suitable for
examining modest jump tables, such as the exception/interrupt vector
pointers of an operating system which may be adjusted at boot up and
when services are registered.  Both static and dynamically linked
programs can be traced using this approach. Statically linked programs
are straightforward as they generally operate in a known address
space, often mapping directly to physical memory. Dynamically linked
programs require the debugger to keep track of memory allocation
operations using either a trace or stop-mode debugging.

\subsection{Sequential Instructions} \label{Sequential Instructions}

For instruction set architectures where all instructions are executed
unconditionally or at least their execution can be determined based on
the program, the instructions between the deltas are assumed to be
executed sequentially. This characteristic means that there is no need
to report them via the trace, only whether the branch was taken or not
and the address of taken indirect jump.

\subsection{Unpredictable PC Discontinuities} \label{unpredpc}

If the program counter is changed by an amount that cannot be
determined from the execution binary, the trace decoder needs to be
given the destination address (i.e. the address of the next valid
instruction).  Examples of this are indirect jumps, where
the next instruction address is determined by the contents of a
register rather than a constant embedded in the source code.

\subsection{Branches} \label{branches}

When a branch occurs, the decoder must be informed of whether it was
taken or not.  For a direct branch, this is sufficient.  There are no
indirect branches in RISC-V; an indirect jump is an unpredictable PC
discontinuity.

\subsection{Interrupts and Exceptions} \label{interruptsexceptions}

Interrupts are a different type of delta, they generally occur
asynchronously to the program's execution rather than intentionally as
a result of a specific instruction or event. Exceptions can be thought
of in the same way, even though they can be typically linked back to a
specific instruction address.  The decoder generally does not know
where an interrupt occurs in the instruction sequence, so the trace
must report the address where normal program flow ceased, as well as
give an indication of the asynchronous destination which may be as
simple as reporting the exception type.  When an interrupt or
exception occurs, or the processor is halted, the final instruction
executed beforehand must be traced.  Following this, for an interrupt
or exception, the next valid instruction address (the first of the
interrupt or exception handler) must be traced in order to instruct the
trace decoder to classify the instruction as an indirect jump even
if it is not.

\subsection{Synchronisation} \label{synchronisation}

In order to make the trace robust there needs to be regular
synchronisation points within the trace. Synchronisation is made by
sending a full valued instruction address (and potentially a context
identifier). The decoder and debugger may also benefit from sending
the reason for synchronising. The frequency of synchronisation is a
trade-off between robustness and trace bandwidth.

The instruction trace encoder needs to synchronise fully:

\begin{itemize}

\item After a reset.
  \item When it becomes qualified\footnote{Qualification is when certain conditions have been met so that a function can be enabled and executed} and instructions have been executed since losing qualification which needed logging by the instruction or data trace encoder.
\item If the instruction is the first of an interrupt service routine or
exception handler (hardware context change).
\item After a prolonged period of time.
\end{itemize}
