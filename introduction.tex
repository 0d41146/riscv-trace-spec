\chapter{Introduction}
\label{sec:intro}

In complex systems understanding program behavior is not easy.
Unsurprisingly in such systems, software sometimes does not behave as
expected. This may be due to a number of factors, for example,
interactions with other cores, software, peripherals, realtime
events, poor implementations or some combination of all of the above.

It is not always possible to use a debugger to observe behavior of a
running system as this is intrusive.  Providing visibility of program
execution is important.  This needs to be done without swamping the
system with vast amounts of data and one method of achieving this is
via Processor Branch Trace.

This works by tracking execution from a known start address and sending
messages about the deltas taken by the program. These deltas are typically
introduced by jump, call, return and branch type instructions,
although interrupts and exceptions are also types of deltas.

Software, known as a decoder, will take this compressed branch trace
and reconstruct the program flow. This can be done off-line or
whilst the system is executing.

In RISC-V, all instructions are executed unconditionally or at least
their execution can be determined based on the program, the
instructions between the deltas are assumed to be executed
sequentially.  This characteristic means that there is no need to
report them via the trace, only whether the branches were taken or not
and the address of taken indirect branches or jumps. If the program
counter is changed by an amount that cannot be determined from the
execution binary, the trace decoder needs to be given the destination
address (i.e. the address of the next valid instruction).  Examples of
this are indirect branches or jumps, where the next instruction
address is determined by the contents of a register rather than a
constant embedded in the source code.

Interrupts generally occur asynchronously to the program's execution
rather than intentionally as a result of a specific instruction or
event.  Exceptions can be thought of in the same way, even though they
can be typically linked back to a specific instruction address.  The
decoder generally does not know where an interrupt occurs in the
instruction sequence, so the trace encoder must report the address
where normal program flow ceased, as well as give an indication of the
asynchronous destination which may be as simple as reporting the
exception type.  When an interrupt or exception occurs, or the
processor is halted, the final instruction executed beforehand must be
traced.


This document serves to specify the ingress port (the signals between
the RISC-V core and the encoder), compressed branch trace algorithm and
the packet format used to encapsulate the compressed branch trace
information.

\subsection{Nomenclature}

In the following sections items in \textbf{font} are signals or
attributes within a packet.

Items in \textit {italics} refer to parameters either built into the
hardware or configurable hardware values.

A decoder is a piece of software that takes the packets emitted by the
encoder and is able to reconstruct the execution flow of the code
executed in the RISC-V core.
