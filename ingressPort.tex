\chapter{Ingress Port} \label{Interface}

\section{Instruction Interface}
This section describes the interface between a RISC-V core and the
trace encoder. The trace interface conveys information about
instruction-retirement and exception events.

Table~\ref{tab:ingress} lists the signals in the interface. The {\bf
  valid} signal is 1 if and only if an instruction retires or traps
(either by generating a synchronous exception or taking an interrupt).
The remaining fields in the packet are only defined when valid is 1.

The {\bf iaddr} field holds the address of the instruction that retired or
trapped. If address translation is enabled, it is a virtual address,
else it is a physical address. Virtual addresses narrower than XLEN
bits are sign-extended, and physical addresses narrower than XLEN bits
are zero-extended.

The {\bf insn} field holds the instruction that retired or
trapped. For instructions narrower than the maximum width, e.g., those
in the RISC-V C extension, the unused high-order bits are
zero-filled. The length of the instruction can be determined by
examining the low-order bits of the instruction, as described in The
RISC-V Instruction Set Manual, Volume I: User-Level ISA, Version 2.1
[1]. The width of the {\bf insn} field, ILEN, is 32 bits for current
implementations.

The {\bf priv} field indicates the privilege mode at the time of instruction
execution. (On an exception, the next valid trace packet's {\bf priv} field
gives the privilege mode of the activated trap handler.) The width of
the {\bf priv} field, PRIVLEN, is 3, and it is encoded as shown in Table 3
2 (TODO: this reference doesn't exist).

The exception field is 0 if this packet corresponds to a retired
instruction, or 1 if it corresponds to an exception or interrupt.  In
the former case, the cause and interrupt fields are undefined and the
tval field is zero.  In the latter case, the fields are set as
follows:

\begin{itemize}
  \item {\bf interrupt} is 0 for synchronous exceptions and 1 for
    interrupts.
  \item {\bf cause} supplies the exception or interrupt cause, as
    would be written to the lower CAUSELEN bits of the mcause CSR. For
    current implementations, CAUSELEN=log2XLEN.
  \item {\bf tval} supplies the associated trap value, e.g., the
    faulting virtual address for address exceptions, as would be
    written to the {\bf mtval} CSR.
\end{itemize}

Future optional extensions may define {\bf tval} to provide ancillary
information in cases where it currently supplies zero.  For cores that
can retire N instructions per clock cycle, this interface is
replicated N times.  Lower- numbered entries correspond to older
instructions.  If fewer than N instructions retire, the valid packets
need not be consecutive, i.e., there may be invalid packets between
two valid packets. If one of the instructions is an exception, no
younger instruction will be valid.

\begin{table}[htp]
    \centering
    \caption{Core-Encoder signals}
    \label{tab:ingress}
    \begin{tabulary}{\textwidth}{|l|l|}
        \hline
        {\bf Signal} & {\bf Function} \\
        \hline
        ivalid & Instruction has retired or trapped (exception) \\
        \hline
        iexception & Exception \\
        \hline
        interrupt & 0 if the exception was synchronous; 1 if interrupt \\
        \hline
        cause [CAUSELEN-1:0] & Exception cause \\
        \hline
        tval[XLEN-1]& Exception data \\
        \hline
        priv[PRIVLEN-1:0] & Privilege mode during execution \\
        \hline
        iaddr [XLEN-1:0] & The address of the instruction \\
        \hline
        instr[ILEN-1:0] & The instruction \\
        \hline
    \end{tabulary}
\end{table}

\subsection {Instruction Opcode at Fetch}

Some cores may not keep hold of the instruction through the pipeline,
in those cases the instruction must be provided at the fetch
stage. Pre-decoded instructions will then be queued through a FIFO in
the encoder for use at retirement.

To support speculative execution, a {\it retract count} input will allow a
specified number of queued instructions to be deleted from the input
side of the queue within the encoder.

The signals in Table~\ref{tab:ingressfetch} will be presented to the
encoder in such systems. Note {\bf fvalid} only applies to the {\bf
  insn}, all other signals in Table~\ref{tab:ingress} are still valid
when {\bf ivalid} is high.

\begin{table}[htp]
    \centering
    \caption{Core-Encoder optional signals}
    \label{tab:ingressfetch}
    \begin{tabulary}{\textwidth}{|l|l|}
        \hline
        {\bf Signal} & {\bf Function} \\
        \hline
        fvalid & Instruction presented is valid at fetch stage \\
        \hline
        retract[RETRACTLEN-1:0] & Number of newest fetched instructions to discard \\
        \hline
    \end{tabulary}
\end{table}

\subsection {Side band signals}

In some circumstances there will be some side band signals which may
affect the encoder's behaviour, for example to start and/or stop
encoding. There will sometimes be cases where the encoder may be
required to affect the behaviour of the core, for example halting.
