\chapter{Ingress Port} \label{Interface}

\section{Interface Requirements}
This section describes in general terms the information which must be passed from the RISC-V core to the trace encoder,
and distinguishes between what is mandatory, and what is optional.

The following information is mandatory:

\begin{itemize}
  \item The number of instructions that are being retired;
  \item Whether there has been an exception or interrupt, and if so the cause (from the \textbf{\textit{ucause/scause/mcause}} CSR)
        and trap value (from the \textbf{\textit{utval/stval/mtval}} CSR);
  \item The current privilege level of the RISC-V core;
  \item The \textit{instruction\_type} of retired instructions for:
    \begin{itemize}
      \item Jumps with a target that cannot be inferred from the source code;
      \item Taken branches;
      \item Return from exception or interrupt (\textbf{\textit{*ret}} instructions).
    \end{itemize}
  \item The \textit{instruction\_address} for:
    \begin{itemize}
      \item Jumps with a target that \textit{cannot} be inferred from the source code;
      \item Taken branches;
      \item The instruction executed immediately after a jump or taken branch (also referred to as the target or destination of the jump or taken branch);
      \item The last instruction executed before an exception or interrupt;
      \item Exceptions;
      \item The first instruction executed following an exception or interrupt;
      \item The last instruction executed before a privilege change;
      \item The first instruction executed following a privilege change;
      \item The first and last instruction being retired.
    \end{itemize}
  \item The number of nontaken branches being retired.
\end{itemize}

The following information is optional:

\begin{itemize}
  \item Context information:
    \begin{itemize}
      \item The context and/or Hart ID;
      \item The type of action to take when context changes.
    \end{itemize}
  \item The \textit{instruction\_type} of instructions for:
    \begin{itemize}
      \item Calls with a target that \textit{cannot} be inferred from the source code;
      \item Calls with a target that \textit{can} be inferred from the source code;
      \item Tail-calls with a target that \textit{cannot} be inferred from the source code;
      \item Tail-calls with a target that \textit{can} be inferred from the source code;
      \item Returns with a target that \textit{cannot} be inferred from the source code;
      \item Returns with a target that \textit{can} be inferred from the source code;
      \item Co-routine swap;
      \item Jumps which don't fit any of the above classifications with a target that \textit{cannot} be inferred from the source code;
      \item Jumps which don't fit any of the above classifications with a target that \textit{can} be inferred from the source code;
      \item Nontaken branches.
    \end{itemize}
  \item If context is supported then the \textit{instruction\_address} for:
    \begin{itemize}
      \item The last instruction executed before a context change;
      \item The first instruction executed following a context change.
    \end{itemize}
\end{itemize}

The mandatory information is the bare-minimum required to implement the branch trace algorithm outlined in Chapter~\ref{Algorithm}.  
The optional information facilitates alternative or improved trace algorithms:

\begin{itemize}
  \item Implicit return mode (see Section~\ref{sec:implicit-return}) requires the encoder to keep track of the number of nested 
    function calls, and to do this it must be aware of all calls and returns regardless of whether the target can be inferred or not;
  \item A simpler algorithm useful for basic code profiling would only report function calls and returns, again 
    regardless of whether the target can be inferred or not;
  \item Branch prediction techniques can be used to further improve the encoder efficiency, particularly for loops 
    (see Section~\ref{sec:branch-prediction}).  This requires the encoder to be aware of the address of all branches, whether
    they are taken or not.
\end{itemize}

\subsection {Jump Classification and Target Inference} \label{Jump Classes}

Jumps are classified as \textit{inferable}, or \textit{uninferable}.  An \textit{inferable} jump has a target which can be
deduced from the binary executable or representation thereof (e.g. ELF).  This means the target of the jump is supplied via

\begin{itemize}
  \item a constant;
  \item a register which contains a constant (e.g. the destination of an \textbf{\textit{lui}} or \textbf{\textit{c.lui}});
  \item a register which contains a constant offset from the PC (e.g. the destination of an \textbf{\textit{auipc}}).
\end{itemize}

Jumps which are not \textit{inferable} are by definition \textit{uninferable}.

Jumps may optionally be further classified according to the recommended calling convention:

\begin{itemize}
  \item \textit{Calls}: 
    \begin{itemize}
      \item \textbf{\textit{jal}} x1;
      \item \textbf{\textit{jal}} x5;
      \item \textbf{\textit{jalr}} x1, rs  where rs != x1;
      \item \textbf{\textit{jalr}} x5, rs  where rs != x5;
      \item \textbf{\textit{c.jalr}} rs1.
    \end{itemize}
  \item \textit{Tail-calls}: 
    \begin{itemize}
      \item \textbf{\textit{jalr}} x0, rs where rs != x1 and rs != x5;
      \item \textbf{\textit{c.jr}} rs1 where rs1 != x1 and rs1 != x5.
    \end{itemize}
  \item \textit{Returns}: 
    \begin{itemize}
      \item \textbf{\textit{jalr}} x0, rs where rs == x1 or rs == x5;
      \item \textbf{\textit{c.jr}} rs1 where rs1 == x1 or rs1 == x5.
    \end{itemize}
  \item \textit{Co-routine swap}: 
    \begin{itemize}
      \item \textbf{\textit{jalr}} x1, x1;
      \item \textbf{\textit{jalr}} x5, x5.
    \end{itemize}
  \item \textit{Other}: 
    \begin{itemize}
      \item \textbf{\textit{jal}} rd where rd != x1 and rd != x5;
      \item \textbf{\textit{jalr}} rd, rs where rd != x0 and rd != x1 and rd != x5.
    \end{itemize}
\end{itemize}

\section{Instruction Interface}
This section describes the interface between a RISC-V core and the
trace encoder that conveys the information described in the previous section.  

\begin{table}[htp]
    \centering
    \caption{Core-Encoder signals - common}
    \label{tab:common-ingress}
    \begin{tabulary}{\textwidth}{|l|p{80mm}|}
        \hline
        \textbf {Signal} & \textbf {Function} \\
        \hline
        \textbf{itype}[\textit{itype\_width\_p}-1:0] & Termination type of the instruction block (see Section~\ref{Jump Classes} for definitions of codes 6 - 15):\newline
        0: Final instruction in the block is none of the other named \textbf{itype} codes;\newline
        1: Exception. An exception occurred following the final retired instruction in the block;\newline
        2: Interrupt. An interrupt occurred following the final retired instruction in the block;\newline
        3: Exception return;\newline
        4: Nontaken branch;\newline
        5: Taken branch;\newline
        6: reserved;\newline
        7: Co-routine swap;\newline
        8: Uninferable call;\newline
        9: Inferrable call;\newline
        10: Uninferable tail-call;\newline
        11: Inferrable tail-call;\newline
        12: Uninferable return;\newline
        13: Inferrable return;\newline
        14: Other uninferable jump;\newline
        15: Other inferable jump.\\
        \hline
        \textbf{cause}[\textit{ecause\_width\_p}-1:0] & Exception or interrupt cause (\textbf{\textit{ucause/scause/mcause}}), \newline
        Ignored unless \textbf {itype}=1 or 2.\\
        \hline
        \textbf{tval}[\textit{iaddress\_width\_p}-1:0]& The associated trap value, e.g. the
    faulting virtual address for address exceptions, as would be
    written to the \textbf{utval/stval/mtval} CSR. Future optional extensions may define \textbf{tval} to provide ancillary information in cases where it currently supplies zero\newline
    Ignored unless \textbf{itype}=1 or 2.\\
        \hline
        \textbf{priv}[\textit{privilege\_width\_p}-1:0] & Privilege level for all instructions in this block.\\
        \hline
        \textbf{context}[\textit{context\_width\_p}-1:0] & Context and/or Hart ID for all instructions in this block.\\
        \hline
        \textbf{iaddr}[\textit{iaddress\_width\_p}-1:0] & The address of the 1st instruction retired in this block.\newline
        Invalid if \textbf{iretire}=0 \\
        \hline
        \textbf{context\_type}[\textit{context\_type\_width\_p}-1:0] & Behavior type of \textbf{context}\newline
        0: Context change with discontinuity;\newline
        1: Precise context change;\newline
        2: Imprecise context change;\newline
        3: Notification.\\
        \hline
    \end{tabulary}
\end{table}

\begin{table}[htp]
    \centering
    \caption{Core-Encoder signals - multiple retirement}
    \label{tab:multi-ingress}
    \begin{tabulary}{\textwidth}{|l|p{80mm}|}
        \hline
        \textbf {Signal} & \textbf {Function} \\
        \hline
        \textbf{iretire}[\textit{iretire\_width\_p}-1:0] & Number of halfwords represented by instructions retired in this block.\\
        \hline
        \textbf{ilastsize}[\textit{ilastsize\_width\_p}-1:0] & The size of the last retired instruction. For cases where the address of the last retired instruction is needed.\\
        \hline
    \end{tabulary}
\end{table}

\begin{table}[htp]
    \centering
    \caption{Core-Encoder signals - multiple non-taken branches}
    \label{tab:multintb-ingress}
    \begin{tabulary}{\textwidth}{|l|p{80mm}|}
        \hline
        \textbf {Signal} & \textbf {Function} \\
        \hline
        \textbf{ntkn}[\textit{ntkn\_width\_p}-1:0] & Number of nontaken branches in this block.\\
        \hline
    \end{tabulary}
\end{table}


\begin{table}[htp]
    \centering
    \caption{Core-Encoder signals - single retirement}
    \label{tab:single-ingress}
    \begin{tabulary}{\textwidth}{|l|p{80mm}|}
        \hline
        \textbf {Signal} & \textbf {Function} \\
        \hline
        \textbf{iretire}[0:0] & Number of instructions retired in this block (0 or 1).\\
        \hline
    \end{tabulary}
\end{table}

Tables~\ref{tab:common-ingress}, ~\ref{tab:multi-ingress} and ~\ref{tab:multintb-ingress}
list the signals in the interface designed to efficiently support retirement of multiple 
instructions per cycle.  The following discussion describes the multiple-retirement behavior.  
However, for cores that can only retire one instruction at a time, the signalling can be 
simplified, and this is discussed subsequently in Section~\ref{sec:single-retire}.  

The information presented on the ingress port represents a contiguous
block of instructions starting at \textbf{iaddr}, all of which retired
in the same cycle. Note if \textbf{itype} is 1 or 2 (indicating an
exception or an interrupt), the number of instructions retired may be
zero. \textbf{cause} and \textbf{tval} are only defined if
\textbf{itype} is 1 or 2. If \textbf{iretire}=0 and \textbf{itype}=0,
the values of all other signals are undefined.

\textbf{iretire} contains the number of half-words represented by
instructions retired in this block, and \textbf{ilastsize} the size of the last
instruction.  Half-words rather than instruction count enables the encoder to easily compute the address of
the last instruction in the block without having access to the size of every 
instruction in the block.  

If address translation is enabled, \textbf{iaddr} is a virtual
address, else it is a physical address. Virtual addresses narrower
than \textit{iaddress\_width\_p} bits must be sign-extended to make
computation of differential addresses easier, and physical addresses
narrower than \textit{iaddress\_width\_p} bits must be zero-extended.

Cores can retire multiple non-taken branches per clock cycle, indicated via \textbf{ntkn}.
However, a consequence of this is that the encoder will be unaware of the addresses of some non-taken branches, 
which will prevent the use of a branch predictor to improve compression (see Section~\ref{sec:branch-prediction}.
For cores that can only retire a maximum of one non-taken branch per clock cycle,
\textbf{ntkn} can be omitted, provided all non-taken branches are indicated via \textbf{itype}.  
The number of non-taken branches is \textbf{ntkn} if \textbf{ntkn} is non-zero, or 1 if \textbf{itype} = 4 and \textbf{ntkn} is zero.
In other words, if for example \textbf{ntkn} is 2 and \textbf{itype} = 4, the encoder will interpret this as 2 non-taken branches,
not 3.

For cores that can retire a maximum of N taken branches per clock
cycle, the signal group (\textbf{iretire}, \textbf{itype},
\textbf{ntkn} (if present), \textbf{ilastsize}, \textbf{iaddr}) must be replicated N times. Signal group 0
represents information about the oldest instruction block, and group N-1
represents the newest instruction block. The interface supports no more
than one privilege, context, exception or interrupt per cycle and so \textbf{priv}, 
\textbf{context}, \textbf{context\_type}, \textbf{cause} and
\textbf{tval} are not replicated. Furthermore, \textbf{itype} can only
take the value 1 or 2 in one of the signal groups, and this must be
the newest valid group (i.e. \textbf{iretire} and \textbf{itype} must
be zero for higher numbered groups). If fewer than N taken branches
are retired in a cycle, then lower numbered groups must be used
first. For example, if there is one taken branch, use only group 0, if
there are two taken branches, instructions up to the 1st taken branch
must be reported in group 0 and instructions up to the 2nd taken branch
must be reported in group 1 and so on.

The \textbf{context} field can be used to convey any additional information to the decoder.  For example:

\begin{itemize}
  \item The Hart ID;
  \item The software thread ID;
  \item It could be used to convey the values of CSRs to the decoder by setting \textbf{context} to the 
    CSR number and value when a CSR is written.
\end{itemize}

Table~\ref{tab:context-type} specifies the actions for the various \textbf{context\_type} values.

\begin{table}[htp]
    \centering
    \caption{Call/return \textbf{context\_type} values and corresponding actions}
    \label{tab:context-type}
    \begin{tabulary}{\textwidth}{|l|l|p{80mm}|}
        \hline
        \textbf {Type} & \textbf {Value} & \textbf {Actions} \\
        \hline
        Context change with discontinuity & 0 &  An example would be a change of Hart.\newline
        Need to report the last instruction executed on the previous context, as well as the 1st on the new context.\newline
        Treated the same as an exception.\\
        \hline
        Precise context change & 1 & Need to output the address of the 1st instruction, and the new context.\newline
        If there were unreported branches beforehand, these need to be output first.\newline
        Treated the same as a privilege change.\\
        \hline
        Imprecise context change & 2 & An example would be a SW thread change.\newline
        Report the new context value at the earliest convenient opportunity.\newline
        It is reported without any address information, and the assumption is that the precise point of context change can be deduced from the source code (e.g. a CSR write). \\
        \hline
        Notification & 3 & An example would be a watchpoint.\newline
        Need to output the address of the watchpoint instruction.\newline
        The context itself is not output.\\
        \hline
    \end{tabulary}
\end{table}

\subsection {Simplifications for single-retirement} \label{sec:single-retire}

For cores that can only retire one instruction at a time, the interface can be simplified to
the signals listed in tables~\ref{tab:common-ingress} and ~\ref{tab:single-ingress}.  The 
simplifications can be summarized as follows:
 
\begin{itemize}
  \item As the number of instructions that are retired in a block is only 0 or 1, the 
    encoder does not need information to enable it to deduce the address of the last instruction 
    retired (it is the same as the 1st and only instruction retired).  So 
    \textbf{ilastsize} is not necessary, and \textbf{iretire} simply indicates whether an
    instruction retired or not;
  \item As the number of non-taken branches retired is never more than 1, and can always
    be indicated via \textbf{itype}, \textbf{ntkn} is not necessary.
\end{itemize}

The parameter \textit{retires\_p} which indicates to the encoder the maximum number of 
instructions that can be retired per cycle can be used by an encoder capable of supporting single or 
multiple retirement to select the appropriate interpretation of \textbf{iretire}.  
\textbf{ilastsize} and \textbf{ntkn} encoder inputs must be tied low when attached to a single-retirement
core that does not provide these outputs.


\subsection {Alternative Multiple-retirement interface configurations}

For a core that can retire multiple instructions per cycle, but no more than one taken branch, the preferred 
solution is to use one of each of the signals from tables~\ref{tab:common-ingress}, ~\ref{tab:multi-ingress} and 
optionally ~\ref{tab:multintb-ingress}.  However, an alternative approach would be to provide explicit details of every 
instruction retired by using N sets of the signal group (\textbf{iretire}, \textbf{itype}) from 
tables~\ref{tab:common-ingress} and ~\ref{tab:single-ingress} with the groups detailing one instruction each 
(replicating the single retirement example N times).  

\subsection {Example Retirement Sequences}

\begin{table}[htp]
    \centering
    \caption{Example 1 : 9 Instructions retired over three cycles, 2 branches} 
    \label{tab:signal-block-9-instructions-2-branches}
    \begin{tabulary}{\textwidth}{|l|l|}
        \hline
        \textbf {Retired} & \textbf {Instruction Trace Block} \\
        \hline
        1000: \textbf{\textit{divuw}} &  \textbf{iretire}=7, \textbf{iaddr}=0x1000, \textbf{ntkn}=0, \textbf{itype}=8\\
        1004: \textbf{\textit{add}} &  \\
        1008: \textbf{\textit{or}} &  \\
        100C: \textbf{\textit{c.jalr}} &  \\
        \hline
        0940: \textbf{\textit{addi}} &  \textbf{iretire}=4, \textbf{iaddr}=0x0940, \textbf{ntkn}=1, \textbf{itype}=5\\
        0944: \textbf{\textit{c.beq}} &  \\
        0946: \textbf{\textit{c.bnez}} &  \\
        \hline
        0988: \textbf{\textit{lbu}} &  \textbf{iretire}=4, \textbf{iaddr}=0x0988, \textbf{ntkn}=0, \textbf{itype}=0\\
        098C: \textbf{\textit{csrrw}} &  \\
        \hline
    \end{tabulary}
\end{table}


\subsection {Sideband signals}

In some circumstances there will be some sideband signals which may
affect the encoder's behavior, for example to start and/or stop
encoding.
There will sometimes be cases where the encoder may be
required to affect the behaviour of the core, for example stalling.

Note, any user defined information that needs to be output by the encoder
will need to be applied to the \textbf{context} value.

\begin{table}[htp]
    \centering
    \caption{User Sideband Encoder Ingress signals}
    \label{tab:ingress-side-band}
    \begin{tabulary}{\textwidth}{|l|l|}
        \hline
        \textbf{Signal} & \textbf{Function} \\
       \hline
        \textbf{user} [\textit{user\_width\_p}-1:0] &  Filtering sideband signals (see Chapter~\ref{ch:filtering})\\
        \hline
        \textbf{halted}& Core is stalled or halted \\
        \hline
        \textbf{reset}& Core in reset \\
        \hline
    \end{tabulary}
\end{table}

\begin{table}[htp]
    \centering
    \caption{User Sideband Encoder Egress signals}
    \label{tab:engress-side-band}
    \begin{tabulary}{\textwidth}{|l|l|}
        \hline
        \textbf{Signal} & \textbf{Function} \\
        \hline
        \textbf{stall}& Stall request to core \\
        \hline
    \end{tabulary}
\end{table}

\subsection {Parameters}

The encoder will have some configurable or variable parameters. Some
of these are related to port widths whilst others may indicate the
presence or otherwise of various features, e.g. filter or comparators.
Table~\ref{tab:parameters} outlines the list of parameters.

How the parameters are input to the encoder is implementation specific. The number range of some of the parameters  may be implementation specific. 

\FloatBarrier
\begin{table}[h]
    \centering
    \caption{Parameters to the encoder}
    \label{tab:parameters}
    \begin{tabulary}{\textwidth}{|l|p{25mm}|p{80mm}|}
        \hline
        \textbf{Parameter name} & \textbf{Range} & \textbf{Description} \\
        \hline
        \textit{bpred\_size\_p} & & Number of entries in the branch predictor is 2\textsuperscript{\textbf{bpred\_size\_p}}. 
                                    Minimum number of entries is 2, so a value of 0 indicates that there is no branch predictor implemented.\\
        \hline
        \textit{call\_counter\_size\_p} &  & Number of bits in the nested call counter is 2\textsuperscript{\textbf{call\_counter\_size\_p}}.
                                    Minimum number of entries is 2, so a value of 0 indicates that there is no implicit return call counter implemented.\\
        \hline
        \textit{context\_type\_width\_p} & 2 & Width of the \textbf{context\_type} bus\\
        \hline
        \textit{context\_width\_p} &  & Width of context bus \\
        \hline
        \textit{ecause\_width\_p} &  & Width of exception cause bus \\
        \hline
        \textit{ecause\_choice\_p} & & Number of bits of exception cause to match using multiple choice \\
        \hline
        \textit{filter\_context\_p} & 0 or 1 & Filtering on context supported when 1 \\
        \hline
        \textit{filter\_ecause\_p} & & Filtering on exception cause supported when non\_zero.  Number of nested exceptions supported is 2\textsuperscript{filter\_ecause\_p} \\
        \hline
        \textit{filter\_interrupt\_p} & 0 or 1 & Filtering on interrupt supported when 1 \\
        \hline
        \textit{filter\_privilege\_p} & 0 or 1 & Filtering on privilege supported when 1 \\
        \hline
        \textit{filter\_tval\_p} & 0 or 1 & Filtering on trap value supported when 1 \\
        \hline
        \textit{iaddress\_lsb\_p} & & LSB of instruction address bus to trace.  1 is compressed instructions are supported, 2 otherwise\\
        \hline
        \textit{iaddress\_width\_p} & & Width of instruction address bus. This is the same as \textit{XLEN}\\
        \hline
        \textit{iretire\_width\_p} & & Width of the \textbf{iretire} bus\\
        \hline
        \textit{ilastsize\_width\_p} & & Width of the \textbf{ilastsize} bus\\
        \hline
        \textit{itype\_width\_p} & & Width of the \textbf{itype} bus\\
        \hline
        \textit{nocontext\_p} & 0 or 1 & Exclude context from \textit{te\_inst} packets if 1 \\
        \hline
        \textit{ntkn\_width\_p} & & Width of the \textbf{ntkn} bus\\
        \hline
        \textit{privilege\_width\_p} & & Width of privilege bus \\
        \hline
        \textit{retires\_p} &  & Maximum number of instructions that can be retired per block \\
        \hline
        \textit{return\_stack\_size\_p} &  & Number of entries in the return address stack is 2\textsuperscript{\textbf{return\_stack\_size\_p}}.
                                    Minimum number of entries is 2, so a value of 0 indicates that there is no implicit return stack implemented.\\
        \hline
        \textit{taken\_branches\_p} & & Number of times \textbf{iretire}, \textbf{itype}, \textbf{ntkn} is replicated\\
        \hline
        \textit{user\_width\_p} & & Width of user-defined filter qualifier input bus \\
        \hline
    \end{tabulary}
\end{table}
\FloatBarrier

\subsection {Discovery of parameter values}

The parameters used by the encoder must be discoverable at
runtime. Some external entity, for example a debugger or a supervisory
hart, would issue a discovery command to the encoder. The encoder will
provide the discovery information as encapsulated in the following
parameters in one or more different formats.  The preferred format
would be in a packet which is sent over the trace infrastructure.

Another format would be allowing the external entity to read the
values from some register or memory mapped space maintained by the
encoder.

\begin{itemize}
    \item \textit{minor\_revision}. Identifies the minor revision.
    \item \textit{version}. Identifies the module version.
    \item \textit{comparators}. The number of comparators is \textit{comparators} + 1.
    \item \textit{filters}. Number of filters is \textit{filters} + 1.
    \item \textit{bpred\_size}. Number of entries in the branch predictor is 2\textsuperscript{\textbf{bpred\_size}}.  No predictor if 0.
    \item \textit{call\_counter\_size}. Width of the nested call counter is 2\textsuperscript{\textbf{call\_counter\_size}}.  No counter if 0.
    \item \textit{context\_type\_width}. Width of the \textbf{context\_type} bus is \textit{context\_type\_width} + 1.
    \item \textit{context\_width}. Width of context input bus is \textit{context\_width} + 1.
    \item \textit{ecause\_choice}. Number of LSBs of the ecause input bus that can be filtered using multiple choice.
    \item \textit{ecause\_width}. Width of the ecause input bus is \textit{ecause\_width} + 1.
    \item \textit{filter\_context}. Filtering on the \textit{context} input bus supported when 1.
    \item \textit{filter\_ecause}. Filtering on the ecause input bus supported when non-zero.  Number of nested exceptions supported is $2^\frac{\textit {filter\_ecause}}{}$.
    \item \textit{filter\_interrupt}. Filtering on the interrupt input signal supported when 1.
    \item \textit{filter\_privilege}. Filtering on the privilege input bus supported when 1.
    \item \textit{filter\_tval}. Filtering on the tval input bus supported when 1.
    \item \textit{iaddress\_lsb}. LSB of iaddress output in trace encoder data messages is \textit{iaddress\_lsb} + 1.
    \item \textit{iaddress\_width}. Width of the iaddress input bus is \textit{iaddress\_width} + 1.
    \item \textit{ilastsize\_width}. Width of the \textbf{ilastsize} bus is \textit{ilastsize\_width} + 1.
    \item \textit{itype\_width}. Width of the \textbf{itype} bus is \textit{itype\_width} + 1.
    \item \textit{iretire\_width}. Width of the \textbf{iretire} bus is \textit{iretire\_width} + 1.
    \item \textit{nocontext}. Context ignored when 1.
    \item \textit{ntkn\_width}. Width of the \textbf{ntkn} bus is \textit{ntkn\_width} + 1.
    \item \textit{privilege\_width}. Width of the privilege input bus is \textit{privilege\_width} + 1.
    \item \textit{retires}. Maximum number of instructions that can be retired per block is \textit{retires} + 1.
    \item \textit{return\_stack\_size}. Number of entries in the return address stack is 2\textsuperscript{\textbf{return\_stack\_size}}. No stack if 0.
    \item \textit{rv32}. ISA is RV32 when 1.
    \item \textit{taken\_branches}. Number of times \textbf{iretire}, \textbf{itype}, \textbf{ntkn} is replicated is \textit{taken\_branches} + 1.
    \item \textit{user\_width}. Width of the \textbf{user} bus is \textit{user\_width} + 1.
\end{itemize}
