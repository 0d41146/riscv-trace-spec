\chapter{Ingress Port} \label{Interface}

\section{Instruction Interface}
This section describes the interface between a RISC-V core and the
trace encoder. The trace interface conveys information about
instruction-retirement and exception events.

\begin{table}[htp]
    \centering
    \caption{Core-Encoder signals}
    \label{tab:ingress}
    \begin{tabulary}{\textwidth}{|l|l|}
        \hline
        \textbf {Signal} & \textbf {Function} \\
        \hline
        \textbf {ivalid} & Instruction has retired or trapped (exception) \\
        \hline
        \textbf {iexception} & Exception \\
        \hline
        \textbf {interrupt} & 0 if the exception was synchronous; 1 if interrupt \\
        \hline
        \textbf {cause} [\textit{context\_width\_p}-1:0] & Exception cause \\
        \hline
        \textbf {tval}[\textit{XLEN}-1]& Exception data \\
        \hline
        \textbf {priv}[\textit{privilege\_width\_p}-1:0] & Privilege mode during execution \\
        \hline
        \textbf {context}[\textit{context\_width\_p}-1:0] & Context and/or Hart ID \\
        \hline
        \textbf {iaddr} [\textit{XLEN}-1:0] & The address of the instruction \\
        \hline
        \textbf{instr}[\textit{insn\_width\_p}-1:0] & The instruction \\
        \hline
    \end{tabulary}
\end{table}

Table~\ref{tab:ingress} lists the signals in the interface. The {\bf
  valid} signal is 1 if and only if an instruction retires or traps
(either by generating a synchronous exception or taking an interrupt).
The remaining fields in the word are only defined when valid is 1.

The \textbf {iaddr} bus holds the address of the instruction that
retired or trapped. If address translation is enabled, it is a virtual
address, else it is a physical address. Virtual addresses narrower
than \textit {XLEN} bits are sign-extended to make computation of differential
addresses easier, and physical addresses narrower than \textit {XLEN} bits are
zero-extended.

The \textbf {instr} bus holds the instruction that retired or
trapped. For instructions narrower than the maximum width, e.g., those
in the RISC-V C extension, the unused high-order bits are
zero-filled. The length of the instruction can be determined by
examining the low-order bits of the instruction, as described in The
RISC-V Instruction Set Manual, Volume I: User-Level ISA, Version 2.1
[1]. The width of the \textbf {insn} bus, \textit {ILEN}, is 32 bits for current
implementations.

The \textbf {priv} bus indicates the privilege mode at the time of instruction
execution. (On an exception, the next valid ingress word's \textbf {priv} bus
gives the privilege mode of the activated trap handler.) The width of
the \textbf {priv} bus, \textit {PRIVLEN}, is typically 3. The actual vaues do not have any special meaning within the encoder other than when filtering. When the level changes the encoder will output the new value in a packet.

The exception bus is 0 if this ingress word corresponds to a retired
instruction, or 1 if it corresponds to an exception or interrupt.  In
the former case, the cause and interrupt buses are undefined and the
tval bus is zero.  In the latter case, the buses are set as
follows:

\begin{itemize}
  \item \textbf {interrupt} is 0 for synchronous exceptions and 1 for
    interrupts.
  \item \textbf {cause} supplies the exception or interrupt cause, as
    would be written to the lower \textit {CAUSELEN} bits of the mcause CSR.
  \item \textbf {tval} supplies the associated trap value, e.g., the
    faulting virtual address for address exceptions, as would be
    written to the \textbf {mtval} CSR. Future optional extensions may define \textbf {tval} to provide ancillary information in cases where it currently supplies zero.  
\end{itemize}

For cores that can retire N instructions per clock cycle, this
interface is replicated N times.  Lower- numbered entries correspond
to older instructions.  If fewer than N instructions retire, the valid
ingress words need not be consecutive, i.e., there may be invalid
ingress words between two valid ingress words. If one of the
instructions is an exception, no younger instruction will be valid.


\subsection {Instruction Opcode at Fetch}

Some cores may not keep hold of the instruction through the pipeline,
in those cases the instruction must be provided at the fetch
stage. Pre-decoded instructions will then be queued through a FIFO in
the encoder for use at retirement.

To support speculative execution, a {\it retract count} input will allow a
specified number of queued instructions to be deleted from the input
side of the queue within the encoder.

The signals in Table~\ref{tab:ingressfetch} will be presented to the
encoder in such systems. Note \textbf {fvalid} only applies to the {\bf
  insn}, all other signals in Table~\ref{tab:ingress} are still valid
when \textbf {ivalid} is high.

\begin{table}[htp]
    \centering
    \caption{Core-Encoder optional signals}
    \label{tab:ingressfetch}
    \begin{tabulary}{\textwidth}{|l|l|}
        \hline
        \textbf {Signal} & \textbf {Function} \\
        \hline
        \textbf {fvalid} & Instruction presented is valid at fetch stage \\
        \hline
        \textbf {retract}[\textit{RETRACTLEN}-1:0] & Number of newest fetched instructions to discard \\
        \hline
    \end{tabulary}
\end{table}

\subsection {Side band signals}

In some circumstances there will be some side band signals which may
affect the encoder's behaviour, for example to start and/or stop
encoding. There will sometimes be cases where the encoder may be
required to affect the behaviour of the core, for example halting.

\begin{table}[htp]
    \centering
    \caption{User Sideband Encoder Ingress signals}
    \label{tab:ingress-side-band}
    \begin{tabulary}{\textwidth}{|l|l|}
        \hline
        \textbf {Signal} & \textbf {Function} \\
       \hline
        \textbf {user} [\textit{user\_width\_p}-1:0] &  Sideband signals \\
        \hline
        \textbf {halted}& Core is halted \\
        \hline
        \textbf {reset}& Core in reset \\
        \hline
    \end{tabulary}
\end{table}

\begin{table}[htp]
    \centering
    \caption{User Sideband Encoder Egress signals}
    \label{tab:ingress-side-band}
    \begin{tabulary}{\textwidth}{|l|l|}
        \hline
        \textbf {Signal} & \textbf {Function} \\
        \hline
        \textbf {halt}& Halt request to core \\
        \hline
    \end{tabulary}
\end{table}


\subsection {Parameters}

The encoder will have some configurable or variable parameters. Some of these are related to port widths whilst others may indicate the presence or otherwise of various feature, e.g. filter or comparitors.

How the parameters are input to the encoder is implementation specific.  

\begin{table}[htp]
    \centering
    \caption{Parameters to the encoder}
    \label{tab:parameters}
    \begin{tabulary}{\textwidth}{|l|p{25mm}|p{80mm}|}
        \hline
        \textbf {Parameter name} & \textbf {Range} & \textbf {Description} \\
        \hline
        \textit {context\_width\_p} & 1-32 & Width of context bus \\
        \hline
        \textit {ecause\_width\_p} & 1-16 & Width of exception cause bus \\
        \hline
        \textit {iaddress\_lsb\_p} & 0-3 & LSB of instruction address bus \\
        \hline
        \textit {iaddress\_width\_p} & 2-128 & Width of instruction address bus. This is the same as \textit {XLEN}\\
        \hline
        \textit {itrace\_fast\_p} & 0 or 1 & Support fast profiling only (no delta mode 1) \\
        \hline
        \textit {nocontext\_p} & 0 or 1 & Ignore context if 1 \\
        \hline
        \textit {notval\_p} & 0 or 1 & Ignore trap value if 1 \\
        \hline
        \textit {privilege\_width\_p} & 1-4 & Width of privilege bus \\
        \hline
        \textit {privilege\_reset\_p} & 0-15 & Default privilege authorization value \\
        \hline
        \textit {ecause\_choice\_p} & 0-6 & Number of bits of exception cause to match using multiple choice \\
        \hline
        \textit {filter\_context\_p} & 0,1 & Filtering on context supported when 1 \\
        \hline
        \textit {filter\_ecause\_p} & 0-15 & Filtering on exception cause supported when non\_zero.  Number of nested exceptions supported is 2\textsuperscript{filter-ecause-p} \\
        \hline
        \textit {filter\_interrupt\_p} & 0,1 & Filtering on interrupt supported when 1 \\
        \hline
        \textit {filter\_privilege\_p} & 0,1 & Filtering on privilege supported when 1 \\
        \hline
        \textit {filter\_tval\_p} & 0,1 & Filtering on trap value supported when 1 \\
        \hline
        \textit {user\_width\_p} & 0-256 & Width of user-defined filter qualifier input bus \\
        \hline
        \textit {retires\_p} & 1-4 & Maximum number of instructions that can retire simultaneously \\
        \hline
        \textit {insn\_width\_p}	& 32, 64, 128 & Width of instruction bus \\
        \hline
    \end{tabulary}
\end{table}

\subsection {Discovery of parameter values}

The parameters used by the encoder must be discoverable at runtime.
