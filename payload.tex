\chapter{Trace Encoder Output Packet}

Used to output encoded instruction trace.  Four different formats are used according to the needs of the encoding algorithm.

\begin{table}[htp]
    \centering
    \caption{Packet Format 0 and 1}
    \label{tab:te_inst0-1}
    \begin{tabulary}{\textwidth}{|l|l|p{100mm}|}
        \hline
        {\bf Field name} & {\bf Bits} & {\bf Description} \\
        \hline
        msg-type & 2 & set to 0x2 \\
        \hline
        format	& 2	& 00 (full-delta): includes branch map and full address \newline
                        01 (diff-delta): includes branch map and differential address\\
        \hline
        branches & 5 & Number of valid bits in branch-map. The length of branch-map is determined as follows: \newline
        0-1: 	1 bit \newline
        2-9: 	9 bits \newline
        10-17: 	17 bits \newline
        18-25: 	25 bits \newline
        26-31: 	31 bits  \\
        \hline
        branch-map & See\footnote{Number of bits determined by branches field} & An array of bits indicating whether branches are taken or not.\newline
        Bit 0 represents the oldest branch instruction executed.   For each bit: \newline
        0: branch taken \newline
        1: branch not taken \\
        \hline
        address	& See\footnote {\label{iaddressbits}Number of bits is iaddress-width-p - iaddress-lsb-p} & Differential or full instruction address, according to format.  \newline
        When branches is 31 (indicating a full address map), the address is only valid if the instruction immediately prior to the final branch causes an unpredicatable discontinuity.\\
        \hline
    \end{tabulary}
\end{table}


\begin{table}[htp]
    \centering
    \caption{Packet Format 2}
    \label{tab:te_inst2}
    \begin{tabulary}{\textwidth}{|l|l|p{100mm}|}
        \hline
        msg-type & 2 & set to 0x2 \\
        \hline
        {\bf Field name} & {\bf Bits} & {\bf Description} \\
        \hline
        format	& 2	& 10 (addr-only): address and no branch map\\
        \hline
        address & See\footnote {\label{addressbitslsbmsb}Total number of bits for address is iaddress-width-p - iaddress-lsb-p.} & Address \newline
        Address is always differential unless full-address is set\\ 
        \hline
    \end{tabulary}
\end{table}

\begin{table}[htp]
    \centering
    \caption{Packet Format 3}
    \label{tab:te_inst3}
    \begin{tabulary}{\textwidth}{|l|l|p{100mm}|}
      \hline
          {\bf Field name} & {\bf Bits} & {\bf Description} \\
          \hline
          msg-type & 2 & Set to 0x2. \\
          \hline
          format & 2 & 11: sync \\
          \hline
          subformat & 2 & Sync sub-format omits fields when not required: \newline
          00 (start): ecause, interrupt and tval omitted \newline
          01 (exception): All fields present (address omitted if implicit-except in set-trace is 1) \newline
          10 (context): address, branch, ecause, interrupt and tval omitted \newline
          11 : reserved \\
          \hline
          context &  See\footnote {\label{contextbits} Total number of bits for contextis context-width-p unless nocontext-p is 1, in which case it is 0} & The instruction context \\
          \hline
          privilege & See\footnote {\label{pivilegewidth} Number of bits is privilege-width-p} & The current privilege level. \\
          \hline
          branch & 1 & If the address points to a branch instruction, set to 1 if the branch was not taken. \newline
          Has no meaning if this instruction is not a branch. \\
          \hline
          address & See\footnote {\label{subformatbits} Number of bits is iaddress-width-p - iaddress-lsb-p, unless subformat is 10, or subformat is 01 and implicit-except in set-trace is 1, in which case the number of bits is 0 (no address).} & Full instruction address.  Address alignment is determined by iaddress-lsb-p. \\
          \hline
          ecause & See\footnote {\label{ecausebits} Number of bits is ecause-width-p if subformat is 01, or 0 otherwise (no exception).} & Exception cause. \\
          \hline
          interrupt & See\footnote {\label{interruptbits}Number of bits is 1 if subformat is 01, or 0 otherwise (no exception).}
 & Interrupt \\
          \hline
          tval & See\footnote {\label{tvalbits} Number of bits is iaddress-width-p if subformat is 01 and notval-p is 0, or 0 otherwise (no exception).}
 & Trap value \\
          \hline
    \end{tabulary}
\end{table}

